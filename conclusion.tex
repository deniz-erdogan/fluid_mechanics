\section{Conclusion}

In this report many different properties and behaviors of fluid flow through an expansion is studied. The flow is simulated for both laminar and turbulent behavior and different expansion ratios and types are analyzed and interpreted. Lastly, an optimized neck for the expansion has been found and tested.\\

\noindent It has been observed that due to turbulent mixing seen in the vicinity of the expansion, these is a loss in the mechanical energy of the fluid which causes a pressure drop on the fluid. This drop has been modeled mathematically and our simulations seemed to be in almost fully agreement with the theoretical values for both pressure drops and velocity profiles.\\

\noindent Furthermore, other properties like the reattachment length of the recirculation zone has been interpreted and there was found a good correlation between our simulation results and experimental results taken from the study of Hammad et al.\cite{hammad_ötügen_arik_1999}. It was surprising how close our results were to both theoretical and experimental data available. \\

\noindent This project was surely a head-first introduction to Computational Fluid Dynamics for me as it was the first time I used these functionalities of ANSYS. While challenging, it was an exciting and fruitful endeavor for me.